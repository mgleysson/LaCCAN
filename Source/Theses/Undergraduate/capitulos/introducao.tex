\mychapter{Introdução}{cap:introducao} \lhead{INTRODUÇÃO}

\section{Motivação}

Ao longo dos últimos anos, dados de sensoriamento remoto, e dados de radar em particular, tornaram-se uma ferramenta essencial para estudos ambientais. Nesse contexto, o sensoriamento remoto por satélite pode fornecer mapeamento em larga escala de áreas impactadas por um desastre, contribuindo significativamente para a consciência situacional, monitorando o evento e os serviços de apoio à decisão. 

De acordo com \citet{Frery99}, um dos maiores desafios hoje em dia é a compreensão precisa do ambiente terrestre, no que diz respeito ao uso da terra, mudanças na cobertura da terra e exploração e conservação dos recursos naturais. Esse conhecimento é essencial para as ações do governo em prol de um desenvolvimento sustentável, o que implica em melhorar a qualidade de vida sem degradar o meio ambiente. Nesse sentido, países, de dimensão continental, como o Brasil necessitam de informações em grande escala as quais, por sua vez, podem ser fornecidas através de sensoriamento remoto. É nesse contexto de âmbito ambiental que as imagens de radar de abertura sintética(SAR) se inserem.

Ainda conforme \citet{Freitas2005}, a década de $90$ foi marcada pela afirmação de imagens de SAR como uma ferramenta para monitoramento da Terra. Vários estudos foram feitos confirmando a relevância dessas imagens, e técnicas específicas de processamento de imagens foram desenvolvidas. Algumas das aplicações de imagens de SAR para monitoramento ambiental são desmatamento e regeneração de florestas secundárias para avaliação do ciclo de carbono, quantificação de biomassa, detecção de petróleo, monitoramento do crescimento de cultivos, previsão de enchentes, vigilância de atividades militares em situações de crise, entre outras aplicações.

Antes de prosseguir sobre a importância das imagens SAR, vale ressaltar a definição do termo RADAR ("RAdio Detection And Ranging") que define um dispositivo capaz de detectar um objeto (alvo), indicando a sua distância (\textit{range}) e sua posição (direção). De acordo com \citet{Pottier2009} algumas das principais vantagens do sensoriamento remoto por RADAR são: (i) pouca dependência das condições atmosféricas e; (ii) possuir sensor ativo, assim, não é afetado pela baixa visibilidade ótica do ambiente imageado. Um sistema SAR possui essas vantagens, visto que que atuam na faixa de microondas e, por isso, é independente da luz solar e de fatores climáticos. Logo, é menos afetado do que os sensores ópticos. 

Visto a relevância do imageamento SAR, ressalta-se a importância da utilização de ferramentas estatísticas para resolver alguns problemas relacionados a imagens devido a sua natureza estocástica. Excelentes resultados frequentemente obtidos com essa abordagem estatística estimularam o desenvolvimento de uma grande quantidade de métodos e técnicas. Nesse contexto, segundo o trabalho de \citet{Gao2010StatisticalMO} a modelagem estatística é essencial para a interpretação de imagens SAR e envolve vários campos, como reconhecimento de padrões, processamento de imagens, análise de sinais, teoria da probabilidade, análise de características de espalhamento eletromagnético de alvos, entre outros. Ainda neste trabalho são discutidos em detalhes vários modelos estatísticos para esse tipo de dado.

Conforme \citet{Mejail2002}, dentre os modelos estatísticos disponíveis, o modelo multiplicativo é o principal deles sendo altamente preciso e bem sucedido, baseando-se no pressuposto de que o campo aleatório observado (retorno) $Z$ é o resultado do produto de dois campos aleatórios independentes e não observados: $X$ e $Y$. Este modelo foi amplamente estudado neste trabalho que visa o desenvolvimento de técnicas de simulação e de estimação de parâmetros no contexto das imagens SAR, tratando especialmente da Lei $G^{0}$ que segue o Modelo Multiplicativo e é um caso especial do modelo $G$ apresentado por \citet{Clutter1997} que provou ser modelo universal interessante para dados SAR, tanto em intensidade quanto em amplitude.

Então este trabalho foca na na distribuição $G_I^0$ (com o subscrito $I$ indicando dados SAR em intensidade) que, como será discutido em detalhes na etapa de Fundamentação, possui seus parâmetros interpretáveis, sendo de bastante relevância conhecê-los para obter características importantes da região imageada, como por exemplo a rugosidade do alvo (textura). Ademais, é extremamente relevante fazer inferências confiáveis sobre o tipo de alvo em análise já que informações visuais muitas vezes não estão disponíveis (por exemplo, áreas sob nebulosidade). Logo, neste trabalho é proposto uma rotina de funções para desempenhar a estimação de parâmetros do modelo $G_I^0$. Além da parte inferencial, este trabalho também propõe contribuições do ponto de vista de simulação, em que funções de visualização de densidade de probabilidade e de geração de variáveis aleatórias $G_I^0$ foram também implementadas. 

Ainda nesse contexto, há dois elementos críticos nessa linha de pesquisa que certamente possuem potencial para dar origem a um bom trabalho:

\begin{itemize}
\item a necessidade de tornar as técnicas acessíveis a usuários não especializados, e
\item a necessidade de otimizar o desenvolvimento de novas técnicas.
\end{itemize}

O primeiro ponto pode ser solucionado por meio do desenvolvimento de métodos que encapsulem diversos algoritmos e forneçam ao usuário, de forma transparente, o retorno da execução com a mínima intervenção possível do mesmo. Já o segundo, consiste em utilizar técnicas de desenvolvimento de software científico.

Logo, é nessa esfera, a dos problemas computacionais que surgem da Modelagem Estatística dos dados SAR (em especial da modelagem utilizando distribuição $G_I^0$) que esse trabalho se insere, tomando como diretrizes os dois elementos críticos acima citados.

\section{Objetivo}

O objetivo geral deste trabalho é propor uma biblioteca de funções para a simulação e inferência considerando a distribuição $G_I^{0}$ como o modelo escolhido para dados SAR. O principal desafio deste trabalho, nesse contexto, consiste de construir uma única rotina que seja capaz de identificar qual é a melhor estratégia de estimação para a amostra de entrada, e que retorne uma estimativa do parâmetro $\theta$ com a menor intervenção possível por parte do usuário.

\section{Solução proposta}

A solução proposta parte do pressuposto que há dados disponíveis em uma amostra $z = (z_1, z_2, \dots, z_n)$ e um modelo para eles $D(\theta)$, com $\theta \in \Theta \subset \mathbb{R}^{p}$ o espaço paramétrico. O modelo $D(\theta)$ induz a única forma de entropia $H$ e, para calcularmos uma estimativa da entropia $H(\theta)$ a partir da amostra $z$ é necessário estimar o parâmetro $\theta$. Como já explicado, o modelo estudado nesse trabalho para dados SAR é proveniente da distribuição $G^0$, proposta por \citet{Clutter1997}. Existem na literatura várias técnicas para estimar $\theta$ a partir da amostra $z$: Momentos fracionários \citep{Mejail2002}, Máxima verossimilhança \citep{FreryMinute2004}, Log-momentos ou Log-cumulantes \citep{krylov2013} \citep{nicolas2002} e estimadores baseados em Distâncias Estocásticas \citep{Cassetti2013} \citep{FreryStochasticDistances2015}. Cada uma dessas técnicas está associada a diferentes algoritmos de estimação, e resulta mais adequada para uma diversidade de situações diferentes (pequenas amostras, textura da região, entre outras). Como já dito, o principal objetivo é implementar uma rotina que, de forma transparante, colete do usuário uma dada amostra, execute a melhor técnica a depender da situação em função apenas dos dados de entrada e retorne a estimativa do parâmetro para o mesmo.


\section{Contribuições}

As contribuições deste trabalho são:

\begin{itemize}
\item A melhor compreensão, por parte do usuário, de informações a respeito da modelagem estatística baseada na distribuição $G_I^0$, trazendo o foco para a parte de simulação e estimação, onde será entregue ao usuário uma biblioteca de funções especializadas em simulação e inferência em modelos para imagens SAR. 
\item A implementação de uma rotina amigável e tolerante a falhas ao usuário com o mínimo de parâmetros de entrada visando a execução do algoritmo de estimação mais apropriado a depender da situação e o posterior retorno do estimador calculado. Tudo de forma a garantir a mínima intervenção possível por parte do usuário.
\end{itemize}

Note que essas contribuições podem facilitar este processo de análise e construção do conhecimento por parte do usuário, tornando tal experiência mais simples e completa, fornecendo para este novas funcionalidades, que com o mínimo de intervenção possível estará apto a utilizar a biblioteca de funções implementadas. Então, vale ressaltar que o foco concentra-se bastanteem integrar as soluções encontradas em ferramentas de uso amigável para esses usuários.

\section{Estrutura do trabalho}

Este trabalho foi dividido em 5 capítulos e um anexo. 
No capítulo~\ref{cap:fundamentacao} introduz algumas das principais teorias, técnicas e análises disponíveis na literatura no contexto da modelagem estatística de dados SAR, simulação e inferência estatística, entre outros tópicos, focando nos conceitos e metodologias aplicados com sucesso em diversos ramos de pesquisa científica.
No capítulo~\ref{cap:metodologia} apresenta a metodologia do trabalho desenvolvido.
No capítulo~\ref{cap:resultados} apresenta os resultados obtidos.
As funções implementadas ao longo do desenvolvimento do projeto se encontram presente no Anexo A.
E, finalmente, no Capítulo~\ref{cap:conclusoes} apresenta as considerações finais, concluindo este trabalho.

\newpage\lhead{\rightmark}
