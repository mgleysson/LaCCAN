\mychapter{Conclusões}{cap:conclusoes}

Neste capítulo serão abordados os avanços no meio científico e a importância proporcionada através do desenvolvimento deste trabalho. Além disso, também serão apresentadas sugestões para trabalhos futuros.

\section{Considerações Finais}

Este trabalho propôs o desenvolvimento de uma biblioteca de funções para simulação e, essencialmente, para inferência estatística em modelos para imagens SAR. O modelo trabalhado para os dados SAR foi dado pela distribuição $G_I^0$. Do ponto de vista de simulação, a principal contribuição se deu por intermédio da implementação de funções de geração de variáveis aleatórias que seguem a distribuição $G_I^0$ e de visualização gráfica da curva de sua função de densidade de probabilidade, como exibido nas figuras \ref{graf_1} e \ref{graf_2} do capítulo \ref{cap:fundamentacao}.

Por outro lado, do ponto de vista da inferência estatística a contribuição relevante deste trabalho se deu por meio da construção de uma rotina de estimação unificada que encapsula um total de quatro algoritmos de estimação de parâmetros (Máxima Verossimilhança, Momentos, Log-Cumulantes e Distâncias Estocásticas) e que segue um fluxograma de execução adaptativo a depender de cada caso que lhe aparece, como mostrado no capítulo anterior. Essa rotina de estimação foi elaborada pensando-se no usuário final, visto que o mesmo pode interagir com a mínima intervenção possível. Essa rotina aproveita o melhor de cada uma das técnicas de estimação implementadas, pois irá executar o algoritmo mais apropriado para cada caso e, dessa forma, a probabilidade de retornar estimativas precisas para o usuário final é bastante elevada.

\section{Trabalhos futuros}

Como sugestões de trabalhos futuros, pode-se pensar em uma análise profunda a cerca da forma de trabalho dos usuários finais e integrar as soluções em ferramentas provendo uma interface mais amigável, como por exemplo, utilizando alguma biblioteca gráfica compatível com a linguagem \texttt{R}.

Ademais, alguns dos objetivos planejados para serem atingidos futuramente com este trabalho estão elencados a seguir:
\begin{itemize}
    \item Implementar mais algoritmos de estimação relevantes existentes na literatura, como por exemplo Métodos Robustos;
    \item Expandir o uso da rotina de estimação a fim de analisar o desempenho, acurácia e robustez dos algoritmos de estimação integrados com dados reais provenientes de imagens SAR disponíveis no site \citet{PoISARpro}, além de dados simulados;
    \item Realizar testes com usuários finais para validação da ferramenta desenvolvida e dos resultados obtidos na prática.
\end{itemize}