\chapter{Manual de utilização das funções desenvolvidas} \label{apendiceA}

\section{Pacotes necessários}

Para que seja possível utilizar plenamente as funções desenvolvidas ao longo deste trabalho será necessário que os seguintes pacotes estejam instalados no ambiente \textit{RStudio}:

\begin{itemize}
    \item stats4
    \item rootSolve
    \item cubature
\end{itemize}

Após a instalação, o usuário pode realizar normalmente as chamadas das funções implementadas.

\section{Principais funções desenvolvidas}

%------------------------------------------------------------------------------------------------------------------

Basicamente três funções principais foram desenvolvidas neste trabalho: \textit{densGI0}, \textit{randGI0} e \textit{estimationGI0}. Não estão listadas aqui as funções auxiliares ou subfunções utilizadas para desenvolvimento dessas funções principais, como, por exemplo, na macro função \textit{estimationGI0} temos basicamente quatro funções auxiliares encapsuladas que são dadas pelas implementações dos respectivos algoritmos de estimação desenvolvidos neste trabalho: \textit{MVestim} (estimação por Máxima Verossimilhança), \textit{MOMestim} (estimação por Momentos), \textit{LCUMestim} (estimação por Log-Cumulantes) e \textit{DTestim} (estimação por Distâncias Estocásticas onde a Distância Triangular foi utilizada).  

Vale ressaltar que todas as funções listadas a seguir possuem tratamento de exceções que possam vir a ocorrer decorrentes de entradas inválidas por parte do usuário. Por exemplo, sabemos que o parâmetro $\alpha$ da $G_I^0$ deve ser negativo, dentre outras restrições.

\newpage

\hrulefill   

\begin{table}[!ht]
\begin{center}
\begin{tabularx}{\textwidth}{ X X}
\hspace{0.5cm} \textbf{densGI0} & \textit{Retorna a densidade de probabilidade da distribuição $G_I^0$}\\
\end{tabularx}
\end{center}
\end{table} 

\vspace{-0.5cm}
\hrulefill  
\vspace{0.5cm}

\textbf{Utilização}

\begin{lstlisting}
   densGI0(x, alpha, gamma, Looks)
\end{lstlisting}

\vspace{0.5cm}

\textbf{Argumentos}

\begin{table}[!ht]
\begin{center}
\begin{tabularx}{\textwidth}{X X}
\hspace{0.5cm} \textit{x} & Vetor contendo os quantis.\\
\hspace{0.5cm} \textit{alpha} & Parâmetro $\alpha$ da distribuição.\\
\hspace{0.5cm} \textit{gamma} & Parâmetro $\gamma$ da distribuição..\\
\hspace{0.5cm} \textit{Looks} & Parâmetro \textit{Looks} da distribuição..\\
\end{tabularx}
\end{center}
\end{table} 

\textbf{Possíveis Exceções}

\vspace{0.5cm}

A mensagem "Entrada Inválida!" será mostrada ao usuário se pelo menos um dos seguintes casos ocorrerem:

\begin{itemize}
    \item Se o parâmetro \textit{x} não for numérico ou contiver algum valor negativo;
    \item Se o parâmetro \textit{alpha} não for numérico ou for maior ou igual a $0$;
    \item Se o parâmetro \textit{gamma} não for numérico ou for menor ou igual a $0$;
    \item Se o parâmetro \textit{Looks} não for numérico ou for menor que $1$.
\end{itemize}

\newpage

%------------------------------------------------------------------------------------------------------------------

\hrulefill   

\begin{table}[!ht]
\begin{center}
\begin{tabularx}{\textwidth}{ X X}
\hspace{0.5cm} \textbf{randGI0} & \textit{Gera variáveis aleatórias $G_I^0$}\\
\end{tabularx}
\end{center}
\end{table} 

\vspace{-0.5cm}
\hrulefill  
\vspace{0.5cm}

\textbf{Utilização}

\begin{lstlisting}
  randGI0(n, alpha, gamma, Looks)
\end{lstlisting}

\vspace{0.5cm}

\textbf{Argumentos}

\begin{table}[!ht]
\begin{center}
\begin{tabularx}{\textwidth}{X X}
\hspace{0.5cm} \textit{n} & Tamanho da amostra (núm. de observações).\\
\hspace{0.5cm} \textit{alpha} & Parâmetro $\alpha$ da distribuição.\\
\hspace{0.5cm} \textit{gamma} & Parâmetro $\gamma$ da distribuição.\\
\hspace{0.5cm} \textit{Looks} & Parâmetro \textit{Looks} da distribuição.\\
\end{tabularx}
\end{center}
\end{table} 

\textbf{Possíveis Exceções}

\vspace{0.5cm}

A mensagem "Entrada Inválida!" será mostrada ao usuário se pelo menos um dos seguintes casos ocorrerem:

\begin{itemize}
    \item Se o parâmetro \textit{n} não for numérico ou for negativo;
    \item Se o parâmetro \textit{alpha} não for numérico ou for maior ou igual a $0$;
    \item Se o parâmetro \textit{gamma} não for numérico ou for menor ou igual a $0$;
    \item Se o parâmetro \textit{Looks} não for numérico ou for menor que $1$.
\end{itemize}

\newpage

%------------------------------------------------------------------------------------------------------------------

\hrulefill   

\begin{table}[!ht]
\begin{center}
\begin{tabularx}{\textwidth}{ X X}
\hspace{0.5cm} \textbf{estimationGI0} & \textit{Rotina de estimação integrada com os algoritmos de estimação}\\
\end{tabularx}
\end{center}
\end{table} 

\vspace{-0.5cm}
\hrulefill  
\vspace{0.5cm}

\textbf{Utilização}

\begin{lstlisting}
   estimationGI0(algorithm, alpha, Looks, sampleSize)
\end{lstlisting}

\vspace{0.5cm}

\textbf{Argumentos}

\begin{table}[!ht]
\begin{center}
\begin{tabularx}{\textwidth}{X X}
\hspace{0.5cm} \textit{algorithm} & Algoritmo de estimação a ser utilizado, cujas opções são NULL, "MV", "MOM", "LCUM" ou "DT". \\
\hspace{0.5cm} \textit{alpha} & Parâmetro $\alpha$ da distribuição. \\
\hspace{0.5cm} \textit{Looks} & Parâmetro $\gamma$ da distribuição. \\
\hspace{0.5cm} \textit{sampleSize} & Parâmetro \textit{Looks} da distribuição. \\
\end{tabularx}
\end{center}
\end{table} 

\textbf{Possíveis Exceções}

\vspace{0.5cm}

A mensagem "Entrada Inválida!" será mostrada ao usuário se pelo menos um dos seguintes casos ocorrerem:

\begin{itemize}
    \item Se o parâmetro \textit{algorithm} for diferente de todas das opções acima;
    \item Se o parâmetro \textit{alpha} não for numérico ou for maior ou igual a $0$;
    \item Se o parâmetro \textit{gamma} não for numérico ou for menor ou igual a $0$;
    \item Se o parâmetro \textit{Looks} não for numérico ou for menor que $1$.
\end{itemize}



