%%%%%%%%%%%%%%%%%%%%%%%%%%%%%%%%%%%%%%%%%
% University Assignment Title Page 
% LaTeX Template
% Version 1.0 (27/12/12)
%
% This template has been downloaded from:
% http://www.LaTeXTemplates.com
%
% Original author:
% WikiBooks (http://en.wikibooks.org/wiki/LaTeX/Title_Creation)
%
% License:
% CC BY-NC-SA 3.0 (http://creativecommons.org/licenses/by-nc-sa/3.0/)
% 
% Instructions for using this template:
% This title page is capable of being compiled as is. This is not useful for 
% including it in another document. To do this, you have two options: 
%
% 1) Copy/paste everything between \begin{document} and \end{document} 
% starting at \begin{titlepage} and paste this into another LaTeX file where you 
% want your title page.
% OR
% 2) Remove everything outside the \begin{titlepage} and \end{titlepage} and 
% move this file to the same directory as the LaTeX file you wish to add it to. 
% Then add \input{./title_page_1.tex} to your LaTeX file where you want your
% title page.
%
%%%%%%%%%%%%%%%%%%%%%%%%%%%%%%%%%%%%%%%%%
%\title{Title page with logo}
%----------------------------------------------------------------------------------------
%	PACKAGES AND OTHER DOCUMENT CONFIGURATIONS
%----------------------------------------------------------------------------------------

\documentclass[12pt]{article}
\usepackage[english]{babel}
\usepackage[utf8x]{inputenc}
\usepackage{amsmath}
\usepackage{graphicx}
\usepackage[colorinlistoftodos]{todonotes}

\begin{document}

\begin{titlepage}

\newcommand{\HRule}{\rule{\linewidth}{0.5mm}} % Defines a new command for the horizontal lines, change thickness here

\center % Center everything on the page
 
%----------------------------------------------------------------------------------------
%	HEADING SECTIONS
%----------------------------------------------------------------------------------------

\textsc{\LARGE Universidade Federal de Alagoas}\\[1.5cm] % Name of your university/college
\textsc{\Large Instituto de Computação (IC)}\\[0.5cm] % Major heading such as course name
\textsc{\large Laboratório de Computação Científica e Análise Numérica (LaCCAN)}\\[0.5cm] % Minor heading such as course title

%----------------------------------------------------------------------------------------
%	TITLE SECTION
%----------------------------------------------------------------------------------------

\HRule \\[0.4cm]
{ \LARGE \bfseries Distribuição Gama}\\[0.4cm] % Title of your document
\HRule \\[1.5cm]
 
%----------------------------------------------------------------------------------------
%	AUTHOR SECTION
%----------------------------------------------------------------------------------------

\begin{minipage}{0.4\textwidth}
\begin{flushleft} \large
\emph{Autor:}\\
Marcos G. S. do Nascimento % Your name
\end{flushleft}
\end{minipage}
~
\begin{minipage}{0.4\textwidth}
\begin{flushright} \large
\emph{Orientador:} \\
Alejandro C. Frery  % Supervisor's Name
\end{flushright}
\end{minipage}\\[2cm]

% If you don't want a supervisor, uncomment the two lines below and remove the section above
%\Large \emph{Author:}\\
%John \textsc{Smith}\\[3cm] % Your name

%----------------------------------------------------------------------------------------
%	DATE SECTION
%----------------------------------------------------------------------------------------

{\large 01/09/2018}\\[2cm] % Date, change the \today to a set date if you want to be precise

%----------------------------------------------------------------------------------------
%	LOGO SECTION
%----------------------------------------------------------------------------------------

% \includegraphics{logo.png}\\[1cm] % Include a department/university logo - this will require the graphicx package
 
%----------------------------------------------------------------------------------------

\vfill % Fill the rest of the page with whitespace

\end{titlepage}


\section{Função densidade de probabilidade}

\textbf{Primeira parametrização:}\\

X $\sim$ \begin{math} \Gamma(\alpha, \beta) \end{math} onde \begin{math} \alpha > 0 \end{math} representa o parâmetro \textit{Shape} e \begin{math} \beta > 0 \end{math} representa o parâmetro \textit{Rate}

\begin{equation}
f(x; \alpha, \beta )=\frac{\beta^{\alpha}x^{\alpha-1}e^{-\beta x}}{\Gamma(\alpha)} \qquad para \quad x > 0
\end{equation}\\

\textbf{Segunda parametrização:}\\

X $\sim$ \begin{math} \Gamma(\alpha, \theta) \end{math} onde \begin{math} \alpha > 0 \end{math} representa o parâmetro \textit{Shape} e \begin{math} \theta > 0 \end{math} representa o parâmetro \textit{Scale} que é avaliado como \begin{math} 1/ \beta \end{math} (inverso do parâmetro \textit{Rate})

\begin{equation}
f(x; \alpha, \theta )=\frac{x^{\alpha-1}e^{\frac{-x}{\theta}}}{\theta^{\alpha}\Gamma(\alpha)} \qquad para \quad x > 0
\end{equation}\\

\textbf{Comparação de ambas:}\\

\begin{equation}
\frac{\beta^{\alpha}x^{\alpha-1}e^{-\beta x}}{\Gamma(\alpha)} \qquad \frac{x^{\alpha-1}e^{\frac{-x}{\theta}}}{\theta^{\alpha}\Gamma(\alpha)}
\end{equation}





\end{document}